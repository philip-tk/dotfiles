\documentclass[12pt, a4paper]{article}
\pagestyle{headings}
\title{My dotfiles}
\author{Philip Thomas K.}

\begin{document}
\maketitle

These are my configuration files.
I will go through them to highlight the settings and keybindings that stray from the default.

\section{Zsh}

\subsection{General}

The shell that I use in my terminal emulator is zsh.
I have enabled vi-mode in zsh.
You can activate it using \texttt{<Esc>}.

I have also enabled extended globbing, which allows me to do things like this:

\begin{center}
\verb|$ rm *~*.txt|
\end{center}

Which removes everything \textbf{except} files ending in ``\texttt{.tex}" .
This is exceptionally helpful when dealing with \LaTeX{} files because the compiler makes all sorts of unnecessary files.

Next up, \texttt{<Ctrl-j>} is used to search through your shell history for the string before your cursor.
This brings up a menu with commands and an index for each command.
Typing in an index will shove that command into your prompt.

I have also enabled reverse incremental search with \texttt{<Ctrl-r>}.
Not only does this search backward for a command, it also accepts pattern searches.
Useful if you are searching for a variation of a common command.

If you are halfway through a command, and wish to use the auto-complete, just hit the \texttt{<Tab>} key.
If there is only one possible auto-complete option, it will go straight into the line.

However, if there are multiple auto-complete options, hitting the \texttt{<Tab>} key lists out the amount of possible options.
Hitting the \texttt{<Tab>} key again will put you in the list.
From here, you can use vim keys to navigate to the correct option and select it by pressing \texttt{<Enter>}.

\subsection{Aliases}

I have about a dozen aliases.
I use most of them, but some of them will obviously change over time.
I use `\texttt{q}' to exit the terminal / shell.
This is because I regularly use vim, and typing out `\texttt{exit}' every time I want to exit is far too tedious.
Next on the list is `\texttt{c}' which clears the terminal.
I use `\texttt{c}' as a prefix for my `\texttt{ls}' based aliases.

My \texttt{ls} based aliases are as follows.
To list out all files and folders that are hidden as well as the non hidden files and folders, I use \texttt{la}.
To list the all hidden and non hidden files and folders and display all their information, I use \texttt{ll}.

Substituting the first `\texttt{l}' in those commands with `\texttt{c}' clears the shell first.

And lastly, I use `\texttt{v}' for vim.

\subsection{Plugins}

I only use two zsh plugins.
\texttt{autojump} and \texttt{zsh-syntax-highlighting}.
The latter is very straightforward, it is just a feature.
The former is very helpful to \texttt{cd} into directories you normally work in.

\end{document}
